%%%%%%%%%%%%%%%%%%%%%%%%%%%%%%%%%%%%%%%%%%%%%%%%%%%%%%%%%%%%%%%%%%%%%%%%%%%%%%%%%%%%%%%%%%%%%
%%									   Conclusion  	    							  	   %%
%%%%%%%%%%%%%%%%%%%%%%%%%%%%%%%%%%%%%%%%%%%%%%%%%%%%%%%%%%%%%%%%%%%%%%%%%%%%%%%%%%%%%%%%%%%%%
\chapter*{Conclusion générale}
\addstarredchapter{Conclusion}
%\addstarredchapter{Introduction générale}
\markboth{CONCLUSION}{}
%ici conclusion
Pour conclure, la construction d'un calendrier potager implique la prise en compte de plusieurs facteurs, parmi lesquels le rythme des phases lunaires est largement utilisé, bien que d'autres rythmes puissent également être explorés. Cependant, pour obtenir un calendrier dynamique et précis, le rythme des phases lunaires s'avère être le plus efficace. En utilisant ce rythme, nous avons pu établir les séquences nécessaires à la création d'un calendrier numérique, offrant ainsi une approche complète et précise pour la planification des cultures.

La technologie Android offre une plate-forme accessible à une large gamme d'utilisateurs, qu'ils soient amateurs, professionnels, étudiants ou enseignants. Sa facilité d'utilisation et la conception intuitive de son interface graphique en font un outil idéal pour la création de diverses applications. Le choix d'Android comme plateforme pour notre application de calendrier potager s'est donc révélé évident. Grâce à cette plateforme, nous avons pu développer une application capable d'estimer les différentes plantes pouvant être cultivées au cours d'un mois donné, facilitant ainsi la planification des cultures pour les utilisateurs.

Cependant, ce projet reste en constante évolution et est susceptible d'évoluer rapidement au fil du temps. En tant que perspectives d'amélioration, l'application pourrait être étendue pour inclure l'utilisation d'une base de données externe, permettant à chaque cultivateur de partager les détails de leur jardin et de leurs expériences de culture. Cela transformerait l'application en une sorte de service web, où chacun pourrait commenter, réagir et partager ses propres connaissances et conseils en matière de jardinage. Cette évolution potentielle permettrait à l'application de devenir une ressource collaborative et communautaire pour les passionnés de jardinage, offrant ainsi une expérience enrichie et plus interactive pour tous les utilisateurs.



