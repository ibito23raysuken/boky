\documentclass[10pt]{article}
\usepackage[utf8]{inputenc}
\usepackage[frenchb]{babel}
\usepackage[T1]{fontenc}
\usepackage{lmodern}
\usepackage{amsfonts}
\usepackage[margin=1in,top=0.5in,headheight=5\baselineskip,headsep=1\baselineskip,includehead]{geometry}
\usepackage{fourier}
\usepackage{tabularx}
\usepackage{multirow}
\usepackage{graphicx}
\usepackage{setspace}
%% some invisible "struts" to help define the structures and row heights.
\newcommand{\aevstrut}{\rule{0pt}{2.9ex}}
\newcommand{\aehstrut}{\rule{0.45em}{0pt}}
\newcolumntype{L}[1]{>{\raggedright\let\newline\\\arraybackslash\hspace{0pt}}m{#1}}
\newcolumntype{C}[1]{>{\centering\let\newline\\\arraybackslash\hspace{0pt}}m{#1}}
\newcolumntype{R}[1]{>{\raggedleft\let\newline\\\arraybackslash\hspace{0pt}}m{#1}}
%% set up and width for the tabularx environment to expand and fit to.
\newlength{\headerwidth}
\setlength{\headerwidth}{\textwidth}
%\setlength{\headheight}{5cm}
\newsavebox{\myheader}
\begin{lrbox}{\myheader}%
	\begin{minipage}[b]{\headerwidth}
		\renewcommand{\arraystretch}{1.29}%
		\begin{tabular}{|C{0.21\headerwidth}|C{0.79\headerwidth}|}\hline
			\multirow{5}{*}{\includegraphics[width=0.21\headerwidth]{espa_new_logo_35.png}} &\textbf{\textsc{Ecole Sup{\'e}rieure Polytechnique d'Antsiranana}}\\
			&\textbf{\textsc{Mention Master STIC}}\\
			&B.P. O 201 $-$ ANTSIRANANA $-$ MADAGASCAR\\
			&\textbf{T{\'e}l.: }+261 (0)32 76 395 40  ou +261 (0)32 03 395 40\\
			&\textbf{Courriel : }mentionsticespa@ gmail.com\\\hline
		\end{tabular}
	\end{minipage}
\end{lrbox}
%% Setting up the header
\usepackage{fancyhdr}
\pagestyle{fancy}
\renewcommand{\headrulewidth}{0pt}
\renewcommand{\footrulewidth}{0pt}
%\lhead{}
\chead{\usebox{\myheader}}
%\rhead{}
%\lfoot{}
%\cfoot{}
%\rfoot{}

%---------------------------------------------------------------------------------------------------------------
%	Debut du document
%---------------------------------------------------------------------------------------------------------------
\begin{document}

%---------------------------------------------------------------------------------------------------------------
%---------------------------------------------------------------------------------------------------------------
%	Premier sujet
%---------------------------------------------------------------------------------------------------------------
%---------------------------------------------------------------------------------------------------------------

%---------------------------------------------------------------------------------------------------------------
%	En tete de chaque page
%---------------------------------------------------------------------------------------------------------------
\begin{tabular}{C{\textwidth}}
	\begin{flushright}
	{\it Maîtriser aujourd'hui la technologie de demain}
	\end{flushright}
	\textbf{\Large Mémoire de fin d'étude $-$ Année Académique 2015/2016}\\\hline
\end{tabular}
\begin{flushright}
{\bf 01 étudiant en Master 2}
\end{flushright}

%---------------------------------------------------------------------------------------------------------------
%	Titre du sujet
%---------------------------------------------------------------------------------------------------------------
\begin{center}
\section*{Titre :  Calendrier potager sous androïde }
\end{center}

%---------------------------------------------------------------------------------------------------------------
%	Objectif
%---------------------------------------------------------------------------------------------------------------
\subsection*{Objectif}
Le calendrier potager est un calendrier de travaux de jardinage qui  aide les agriculteurs à
mieux récolter sachant que chaque espèce potagère demande des soins spécifiques. Le calendrier
vas donc les accompagner mois par mois pour savoir : quelle variété choisir et quand planter et
quand récolter, quand effectuer les semis du potager, l'éclaircissage et le repiquage. C'est donc un
tableau de bord pour des cultures optimisées. \\
L’objectif de ce travail est donc créer un calendrier interactif de potager. \\
 
OBJECTIF : Conception et réalisation d’un calendrier potager sous androïde  

%---------------------------------------------------------------------------------------------------------------
%	Travaux demandes
%---------------------------------------------------------------------------------------------------------------
\subsection*{Travaux demand{\'e}s}
\begin{itemize}
	\item Etude bibliographique
	\item Modélisation de l’application
	\item Conception et développement de l’application
	\item Implémentation et test 
\end{itemize}

%---------------------------------------------------------------------------------------------------------------
%	Documentation
%---------------------------------------------------------------------------------------------------------------
%\subsection*{Bibliographie}
%$[1]$ Pramod Pal, TM Shuhum, Amit Ojha. (2014). Simulation of Brushless DC Motor for Performance Analysing using Matlab/Simulink Environment. International Journal on Recent and Innovation Trends in Computing and Communication. Volume 2 (Numéro 6), 1564-1567.\\

%$[2]$ M.ATTOU Amine (2011), Commande par mode glissant de la machine synchrone à aimants permanents, Master en Électrotechnique, UNIVERSITE DJILLALI LIABES DE SIDI BEL-ABBES.\\

%$[3]$ Jérôme FAUCHER, (2006). Les plans d'expériences pour le réglage de commande à base de logique floue. Thèse de doctorant de l'Institut National Polytechnique de Toulouse.

%---------------------------------------------------------------------------------------------------------------
%	Lieu de travail
%---------------------------------------------------------------------------------------------------------------
%\subsection*{Lieu de travail}
%Laboratoire d'Automatique \& Laboratoire {\'E}lectronique Industrielle

%---------------------------------------------------------------------------------------------------------------
%	Encadreur
%---------------------------------------------------------------------------------------------------------------
\subsection*{Encadreur(s)}
\begin{itemize}
	\item ANDRIANAJAINA Todizara%, Professeur (razafinjaka@yahoo.fr)
	\item ANDRIAMIHARINJAKA Hasina , Docteur %(philibert.andrianiriniaimalaza@gmail.com
\end{itemize}
\end{document}