%%%%%%%%%%%%%%%%%%%%%%%%%%%%%%%%%%%%%%%%%%%%%%%%%%%%%%%%%%%%%%%%%%%%%%%%%%%%%%%%%%%%%%%%%%%%%
%%									   Introduction		    							   %%
%%%%%%%%%%%%%%%%%%%%%%%%%%%%%%%%%%%%%%%%%%%%%%%%%%%%%%%%%%%%%%%%%%%%%%%%%%%%%%%%%%%%%%%%%%%%%
\chapter*{Introduction}
\addstarredchapter{Introduction}
%\addstarredchapter{Introduction générale}
\markboth{INTRODUCTION}{}
Le domaine de l'agriculture a toujours été fortement influencé par des facteurs naturels tels que le climat et les cycles de la nature. Parmi ces influences, les phases lunaires ont depuis longtemps captivé l'attention des agriculteurs en raison de leur prétendu impact sur la croissance des cultures. En réponse à cette fascination, notre mémoire se penche sur le développement d'une application Android révolutionnaire conçue pour les agriculteurs modernes, qui vise à exploiter la sagesse séculaire des phases lunaires pour optimiser leurs calendriers de plantation et de culture. Cette application, baptisée "Calendrier de Potager" se positionne comme un outil novateur pour les agriculteurs soucieux d'améliorer leur rendement et leur durabilité, en harmonie avec les cycles de la nature.\\
Notre mémoire se consacrera à l'analyse approfondie de cette application, de sa conception à son développement, en passant par son utilité et ses implications pour l'agriculture contemporaine. Nous examinerons également les principaux concepts et théories qui sous-tendent l'influence des phases lunaires sur la croissance des cultures, ainsi que les méthodes scientifiques utilisées pour valider de telles affirmations. \\
%ettofement 
À travers ce travail, nous aspirons à fournir un aperçu complet de l'application "Calendrier de Potager" en mettant en lumière son potentiel pour révolutionner les pratiques agricoles et améliorer la durabilité de l'agriculture. Cette mémoire servira de ressource précieuse pour les agriculteurs, les chercheurs et les passionnés de l'agriculture, en explorant comment la technologie moderne peut s'allier à la sagesse traditionnelle pour favoriser une agriculture plus éclairée et respectueuse de l'environnement. \\
En même temps, le domaine de la téléphonie mobile évolue constamment depuis son avènement. Ainsi, le sujet propose d'allier ces deux aspects. Il s'agit de construire un calendrier potager pour Android afin de déterminer les cultures d'un jardin pour une date déterminée. Ainsi, nous aborderons différents aspects et concepts pour développer notre application, notamment :\\
\begin{itemize}
 \item[•]Premièrement, de savoir ce qu’est vraiment une application Android.
 \item[•]Deuxièmes on parlera de notre environnement qui est Android.
 \item[•]Et enfin, troisièmes on étudiera comment  modéliser cette application et de le générer.
\end{itemize}
%\blindtext \\
%\linebreak
%\blindtext \\
%\linebreak
%\blindtext \\


