%%%%%%%%%%%%%%%%%%%%%%%%%%%%%%%%%%%%%%%%%%%%%%%%%%%%%%%%%%%%%%%%%%%%%%%%%%%%%%%%%%%%%%%%%%%%%
%%									ANNEXES 												%
%%%%%%%%%%%%%%%%%%%%%%%%%%%%%%%%%%%%%%%%%%%%%%%%%%%%%%%%%%%%%%%%%%%%%%%%%%%%%%%%%%%%%%%%%%%%%
\chapter{Annexes}

%%%%%%%%%%%%%%%%%%%%%%%%%%%%%%%%%%%%%%%%%%%%%%%%%%%%%%%%%%%%%%%%%%%%%%%%%%%%%%%%%%%%%%%%%%%%%


\section{CODE SOURCE}
	%\blindtext
	%\linebreak
	%\blindtext
	\subsection{extrait de modèle de base de donnée}
		\begin{lstlisting}
   %code source
 * Created by ibito-PC on 28/05/2017.
 */
public class Model extends SQLiteOpenHelper {
    public static final String POTAGER_KEY = "id";
    public static final String POTAGER_NOM = "nom";
    public static final String POTAGER_PLANTER = "plante";
    public static final String POTAGER_TABLE_NAME = "potager";

    public static final String POTAGER_TABLE_CREATE = 
    "CREATE TABLE " + POTAGER_TABLE_NAME + "(" + POTAGER_KEY + 
    " INTEGER PRIMARY KEY AUTOINCREMENT," +
            POTAGER_NOM + " TEXT, " + POTAGER_PLANTER + " TEXT);";
    public static final String POTAGER_TABLE_DROP = 
    "DROP TABLE IF EXISTS " + POTAGER_TABLE_NAME + ";";
    public static final String POTAGER_SELECT = "SELECT" +
     "*" + " FROM " + POTAGER_TABLE_NAME + ";";
    String selectQuery = "Select name from sqlite_master where type ='table'";

    public Model(Context context, String name, 
    SQLiteDatabase.CursorFactory factory, int version) {
        super(context, name, factory, version);
    }

    @Override
    public void onCreate(SQLiteDatabase db) {
        db.execSQL(POTAGER_TABLE_CREATE);
    }

    @Override
        public void onUpgrade(SQLiteDatabase db, 
        int oldVersion, int newVersion) {
        db.execSQL(POTAGER_TABLE_DROP);
        onCreate(db);
    }



    public void Sauvegarde(Monpotager m) {
        ContentValues value = new ContentValues();
        value.put(POTAGER_NOM, m.getNom());
        value.put(POTAGER_PLANTER, m.getPlante());
        Long id = m.getId();
        if (id == 0) {

            id = this.getWritableDatabase().
            insert(POTAGER_TABLE_NAME, null, value);
            m.setId(id);

        } else {
            this.getWritableDatabase().update(POTAGER_TABLE_NAME, 
            value, POTAGER_KEY + "= ?", new String[]{String.valueOf(id)});
        }

    }

    public void effacertout() {
        this.getWritableDatabase().delete(POTAGER_TABLE_NAME, 
        null, null);
    }

    public void effacerPotager(Monpotager p) {
        this.getWritableDatabase().delete(POTAGER_TABLE_NAME, 
        POTAGER_KEY + "= ?", new String[]{String.valueOf(p.getId())});
    }

    public Cursor getAllpotagerAscursor() {
        return this.getReadableDatabase().rawQuery(POTAGER_SELECT, null);
    }
    public List<String> allTableName() {
        List<String> result = new ArrayList<String>();
        Cursor c = this.getReadableDatabase().rawQuery(selectQuery, null);
        if (c.moveToFirst()) {
            do {
                String n = c.getString(c.getColumnIndex("name"));
                result.add(n);
            } while (c.moveToNext());
        }
        return result;
    }
    public List<Monpotager> getAllpotager() {
        List<Monpotager> resulat = new ArrayList<Monpotager>();
        Cursor c = this.getAllpotagerAscursor();
        /*
        if (c.moveToFirst()) {
            do {
                Long id = c.getLong(c.getColumnIndex(POTAGER_KEY));
                p.setId(id);
                p.setNom(c.getString(c.getColumnIndex(POTAGER_NOM)));
                p.setPlante(c.getString(c.getColumnIndex(POTAGER_PLANTER)));
                resulat.add(p);
            } while (c.moveToNext());
        }
      
        if (c.getCount() == 0) {
            return null;
        } else {
            c.moveToFirst();
            while (c.isAfterLast() == false) {
                Monpotager p = new Monpotager();
                Long id = c.getLong(c.getColumnIndex(POTAGER_KEY));
                p.setId(id);
                p.setNom(c.getString(c.getColumnIndex(POTAGER_NOM)));
                p.setPlante(c.getString(c.getColumnIndex(POTAGER_PLANTER)));
                Log.i("test_BD","liste"+c.getString(c.getColumnIndex(POTAGER_NOM)));
                resulat.add(p);
                c.moveToNext();
            }
            c.close();
            return resulat;
        }
    }
}
		\end{lstlisting}
\subsection{extrait de code pour lister les potager}
\begin{lstlisting}

/**
 * Created by ibito-PC on 25/05/2017.
 */
public class Liste_potagerActivity extends AppCompatActivity {
    //liste
    ArrayList liste_potager;
    private ListView liste = null;
    ArrayAdapter<Potager> adapter;
    Resources resources_potager;
    final String EXTRA_POTAGE="potage";
    public Liste_potagerActivity(){
    }

    public void onCreate(Bundle savedInstanceState){
        super.onCreate(savedInstanceState);
        //setContentView(R.layout.activity_liste);
        //creation pop up
        AlertDialog.Builder alertDialog = 
        new AlertDialog.Builder(Liste_potagerActivity.this);
        LayoutInflater inflater = getLayoutInflater();
        View convertView = (View) inflater.inflate(R.layout.custom, null);
        alertDialog.setView(convertView);
        alertDialog.setTitle("List de potage");

        //lecture de resource
        resources_potager =getResources();
        final String[] nom_potager = resources_potager.getStringArray(R.array.potage);
        // recopie dans le ArrayList
        liste_potager = new ArrayList<Potager>();
        for (int   i=0; i<nom_potager.length; ++i) {
            liste_potager.add(new Potager(nom_potager[i]));
        }
        //adapteur
        adapter = new ArrayAdapter<Potager>(this,
        android.R.layout.simple_list_item_1, liste_potager);
        //liste= (ListView) findViewById(R.id.listView2);


        ListView lv = (ListView) convertView.findViewById(R.id.listView1);
        lv.setAdapter(adapter);
        lv.setOnItemClickListener(new AdapterView.OnItemClickListener()
                                             {

                     @Override
                      public void onItemClick(AdapterView<?> parent, View view, int position,
                                                                         long id) {
                        // String content = (String)parent.
                        getItemAtPosition(position);
                            // Log.i("potage","----------------------------");
                       Intent intent=new Intent(Liste_potagerActivity.this,
                       Choix_potageActicity.class);
                      intent.putExtra(EXTRA_POTAGE,liste_potager.get(position).toString());
                      startActivity(intent);
                     finish();

                                                 }

                                             });
        alertDialog.show();
    }
    public void onListItemClick(ListView l, View v, int position, long id) {

    }
public void onBackPressed() {
    super.onBackPressed();  
    finish();
}

}
\end{lstlisting}
	%\blindtext
	\subsection{calcule de Phase Lunaire}
		\begin{lstlisting}

/**
 * Created by ibito-PC on 07/05/2017.
 */
public class Calcul{
    private double NL,PL,PQ,DQ;
    private int nomMois,nomAnnee;
    private  boolean resultat;
    private ArrayList<Mois> moisListe;
    int m;

    public Calcul(int mois,int annee,Context context) {
        int [] nJour= context.getResources().getIntArray(R.array.nJour);
        String[] noms = context.getResources().getStringArray(R.array.noms);
            nomMois=mois;
            nomAnnee=annee;
            resultat=bissextille(nomAnnee);
            //creation du liste
            moisListe = new ArrayList<Mois>();

            for (int i=0; i<noms.length; ++i) {
                moisListe.add(new Mois(noms[i],nJour[i],i));
            }
            //changement de fevrier
            if(nomMois==2 && resultat==true){
                moisListe.remove(1);
                moisListe.add(1,new Mois("Fevrier",29,1));
            }
            else{
                moisListe.remove(1);
                moisListe.add(1,new Mois("Fevrier",28,1));
            }
        NL=calcul(mois,resultat,nomAnnee,"NL");
        PL=calcul(mois,resultat,nomAnnee,"PL");
        PQ=calcul(mois,resultat,nomAnnee,"PQ");
        DQ=calcul(mois,resultat,nomAnnee,"DQ");
    }
    private double calcul(int m,boolean bis,int annee,String phase_lune){
        float k,nbrjour_annee,somme=0,T,E,M,M_second,F,n,somme_cor,somme_cor2,Z;
        float[] tab_cor=new float[26];
        double JD,JDE,reste;
        int alpha,A,B,C,D,E_final,reponse;

        for(int i=m;i>=0;--i)
            somme=somme+moisListe.get(i).getNbrJour();
        if(bis==true)
            nbrjour_annee=366;
        else
            nbrjour_annee=365;
        somme=somme/nbrjour_annee;
        somme=(float)arrondi(somme,2);
        somme=somme+annee;
        k=calcul_k(somme,phase_lune);
        T=(float)arrondi(k/1236.85,5);
        JDE=(double)(2451550.09765+(29.530588853*k)+(0.0001337*T*T)-(0.00000015*T*T*T)+(0.00000000073*T*T*T*T));
        E=(float)(1-(0.002516*T)-(0.0000074*T*T));
        M=(float)(2.5534 + (29.10535669*k)-(0.0000218*T*T)- (0.00000011*T*T*T));
        M_second=(float)(201.5643 + (385.81693528 *(k)) + (0.0107438*T*T) + (0.00001239*T*T*T) - (0.000000058*T*T*T*T));
        F=(float)(160.7108 + 390.67050274*(k) - 0.0016341*T*T - 0.00000227*T*T*T + 0.000000011*T*T*T*T);
        n= (float)(124.7746 - 1.5637558*(k) + 0.0020691*T*T + 0.00000215 *T*T*T);

        tab_cor[0]=(float)(299.77 + 0.107408*k - 0.009173*T*T);
        tab_cor[1]= (float)(251.88 + 0.016321*k);
        tab_cor[2]=(float)(251.83 + 26.651886*(float)k);
        tab_cor[3]=(float)(349.42 + 36.412478*k);
        tab_cor[4]=(float)( 84.66 + 18.206239*k);
        tab_cor[5]=(float)(141.74 + 53.303771*k);
        tab_cor[6]=(float)(207.14 + 2.453732*k);
        tab_cor[7]=(float)(154.84 + 7.30686*k);
        tab_cor[8]=(float)(34.52 + 27.261239*k);
        tab_cor[9]=(float)(207.19 + 0.121824 *k);
        tab_cor[10]=(float)(291.34 + 1.844379 *k);
        tab_cor[11]=(float)(161.72 + 24.198154*k);
        tab_cor[12]=(float)(239.56 + 25.513099 *k);
        tab_cor[13]=(float)(331.55 + 3.592518*k);

        tab_cor[0]=(float)(0.000325*Math.sin(Math.toRadians(tab_cor[0])));
        tab_cor[1]=(float)(0.000165*Math.sin(Math.toRadians(tab_cor[1])));
        tab_cor[2]=(float)(0.000164*Math.sin(Math.toRadians(tab_cor[2])));
        tab_cor[3]=(float)(0.000126*Math.sin(Math.toRadians(tab_cor[3])));
        tab_cor[4]=(float)(0.00011*Math.sin(Math.toRadians(tab_cor[4])));
        tab_cor[5]=(float)(0.000062*Math.sin(Math.toRadians(tab_cor[5])));
        tab_cor[6]=(float)(0.00006*Math.sin(Math.toRadians(tab_cor[6])));
        tab_cor[7]=(float)(0.000056*Math.sin(Math.toRadians(tab_cor[7])));
        tab_cor[8]=(float)(0.000047*Math.sin(Math.toRadians(tab_cor[8])));
        tab_cor[9]=(float)(0.000042*Math.sin(Math.toRadians(tab_cor[9])));
        tab_cor[10]=(float)(0.00004*Math.sin(Math.toRadians(tab_cor[10])));
        tab_cor[11]=(float)(0.000037*Math.sin(Math.toRadians(tab_cor[11])));
        tab_cor[12]=(float)(0.000035*Math.sin(Math.toRadians(tab_cor[12])));
        tab_cor[13]=(float)(0.000023*Math.sin(Math.toRadians(tab_cor[13])));

        somme_cor=0;
        for(int i=0;i<=13;++i)
            somme_cor=somme_cor+tab_cor[i];

        tab_cor[0]=(float)(-0.4072*Math.sin(Math.toRadians(M_second)));
        tab_cor[1]=(float)(0.17241*E*Math.sin(Math.toRadians(M)));
        tab_cor[2]=(float)(0.01608*Math.sin(Math.toRadians(2*M_second)));
        tab_cor[3]=(float)(0.01039*Math.sin(Math.toRadians(2*F)));
        tab_cor[4]=(float)(0.00739*E*Math.sin(Math.toRadians(M_second-M)));
        tab_cor[5]=(float)(-0.00514*E*Math.sin(Math.toRadians(M_second+M)));
        tab_cor[6]=(float)(0.00208*E*E*Math.sin(Math.toRadians(2*M)));
        tab_cor[7]=(float)(-0.00111*Math.sin(Math.toRadians(M_second-2*F)));
        tab_cor[8]=(float)(-0.00057*Math.sin(Math.toRadians(M_second+2*F)));
        tab_cor[9]=(float)(0.00056 *Math.sin(Math.toRadians(2*M_second+M)));
        tab_cor[10]=(float)(-0.00042*Math.sin(Math.toRadians(3*M_second)));
        tab_cor[11]=(float)(0.00042 *Math.sin(Math.toRadians(M+2*F)));
        tab_cor[12]=(float)(0.00038*E*Math.sin(Math.toRadians(M-2*F)));
        tab_cor[13]=(float)(-0.00024*E*Math.sin(Math.toRadians(2*M_second-M)));
        tab_cor[14]=(float)(-0.00017*Math.sin(Math.toRadians(n)));
        tab_cor[15]=(float)(-0.00007*Math.sin(Math.toRadians(M_second-2*M)));
        tab_cor[16]=(float)(0.00004*Math.sin(Math.toRadians(2*M_second-2*F)));
        tab_cor[17]=(float)(0.00004*Math.sin(Math.toRadians(3*M)));
        tab_cor[18]=(float)(0.00003*Math.sin(Math.toRadians(M_second+M-2*F)));
        tab_cor[19]=(float)(0.00003*Math.sin(Math.toRadians(2*M_second+2*F)));
        tab_cor[20]=(float)(-0.00003*Math.sin(Math.toRadians(M_second+M+2*F)));
        tab_cor[21]=(float)(0.00003*Math.sin(Math.toRadians(M_second-M+2*F)));
        tab_cor[22]=(float)(-0.00002*Math.sin(Math.toRadians(M_second-M-2*F)));
        tab_cor[23]=(float)(-0.00002*Math.sin(Math.toRadians(3*M_second-M)));
        tab_cor[24]=(float)(0.00002*Math.sin(Math.toRadians(4*M_second)));

        somme_cor2=0;
        for(int i=0;i<=24;++i)
            somme_cor2=somme_cor2+tab_cor[i];

        JD=JDE+somme_cor2+somme_cor;
        JD=JD+0.5;
        Z=(int)JD;
        reste=JD-Z;
        reste=arrondi(reste,5);
        alpha=(int)((Z-1867216.25)/36524.25);
        A=(int)(Z+1+alpha-(alpha/4));
        B=A+1524;
        C=(int)((B-122.1)/365.25);
        D=(int)(365.25*C);
        E_final=(int)((B-D)/30.6001);
        reponse=(B-D-(int)(30.6001*E_final)+(int)reste);
        return reponse;
    }

    public static double arrondi(double A, int B) {
        return (double) ( (int) (A * Math.pow(10, B) + .5)) / Math.pow(10, B);
    }
    public static float calcul_k(float somme,String phase_lune){
        float k=0;
        k=(int)((somme-2000)*12.3685);
        if(phase_lune=="NL")
            k=(float)(k+0);
        else if(phase_lune=="PQ"){
            k=(float)(k+0.25);
        }
        else if(phase_lune=="DQ")
            k=(float)(k+0.75);
        else
            k=(float)(k+0.5);

        return k;
    }
    //bisextille
    private boolean bissextille(int annee){
        boolean resultat;
        if((annee%400)==0||((annee%100!=0)&&(annee%4==0)))
            resultat=true;
        else
            resultat=false;
        return  resultat;
    }


}

		\end{lstlisting}
	\subsection{choix des potager}
		\begin{lstlisting}
		package calendrierpotager.ibito.com.calendrierpotager;

import android.content.Context;
import android.content.Intent;
import android.content.SharedPreferences;
import android.os.Bundle;
import android.preference.PreferenceManager;
import android.support.v7.app.AppCompatActivity;
import android.util.Log;
import android.widget.CalendarView;
import android.widget.TextView;
import android.widget.Toast;

/**
 * Created by ibito-PC on 19/04/2017.
 */

public class CalendrierActivity extends AppCompatActivity implements CalendarView.OnDateChangeListener {
    final String EXTRA_JOUR="jour";
    final String EXTRA_MOIS="mois";
    final String EXTRA_ANNEE="annee";
    final String TYPE_PLANTE="type_plante";
    //sesion
    String myString,typepotage;
    Calcul cal_calend;
    int jour, mois, annee, NL, PL, DQ, PQ;
    public static final String MyPREFERENCES = "Mon_reference" ;
    public static final String Name = "nameKey";

        protected void onCreate(Bundle savedInstanceState) {
        super.onCreate(savedInstanceState);
        setContentView(R.layout.activity_calendrier);
        //session
        SharedPreferences preferences =  getSharedPreferences(MyPREFERENCES, 0);
        myString = preferences.getString(Name,null);
        //Log.i("information","--------------------------------------------------------------------------------"+myString);
        //afichage textView5
        TextView t1=(TextView) findViewById(R.id.textView7);
        TextView t2 = (TextView) findViewById(R.id.textView5);
        if (myString==null){
            t1.setText("");
            t2.setText("");
        }
        else{
            t2.setText(myString);
        }
        CalendarView calendrier =(CalendarView) findViewById(R.id.calendarView);
        calendrier.setOnDateChangeListener(this);
    }

    @Override
            public void onSelectedDayChange(CalendarView view, int year, int month, int dayOfMonth) {
            if (myString==null){

                cal_calend = new Calcul(month, year, getApplicationContext());
                NL = (int) cal_calend.getNL();
                PL = (int) cal_calend.getPL();
                PQ = (int) cal_calend.getDQ();
                DQ = (int) cal_calend.getPQ();
                if(dayOfMonth<PQ)
                    typepotage="feuille";
                else if(dayOfMonth>PQ && dayOfMonth<NL)
                    typepotage="fleur";
                else if(dayOfMonth>PQ && dayOfMonth<NL)
                    typepotage="fruit";
                 else
                    typepotage="racine";

           // Liste_potagerActivity l=new Liste_potagerActivity();
            Intent intent=new Intent(CalendrierActivity.this,Liste_potagerActivity.class);
            intent.putExtra(TYPE_PLANTE,typepotage);
            this.startActivity(intent);
        }
        else{
            Intent intent=new Intent(CalendrierActivity.this,Choix_potageActicity.class);
            intent.putExtra(EXTRA_JOUR,""+dayOfMonth);
            intent.putExtra(EXTRA_MOIS,""+month);
            intent.putExtra(EXTRA_ANNEE,""+year);
            Toast.makeText(CalendrierActivity.this,"jour"+dayOfMonth,Toast.LENGTH_SHORT).show();
            this.startActivity(intent);
        }

    }
    public void onBackPressed() {
        SharedPreferences sharedpreferences = getSharedPreferences(CalendrierActivity.MyPREFERENCES, Context.MODE_PRIVATE);
        SharedPreferences.Editor editor = sharedpreferences.edit();
        editor.clear();
        editor.commit();
        super.onBackPressed();
        finish();
    }
}

		\end{lstlisting}