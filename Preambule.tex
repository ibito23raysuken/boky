%% Copyright (C) 2016 Ecole Supérieur Polytechnique d'Antsiranana
%%
%% Ce document était rédiger pour servir de modèle de rédaction de
%% rapport de mémoire pour l'ESPA. Il est sous licence libre.
%%
%% le concepteur et celui qui le met ajours est actuellement
%% Mr ANDRIANAJAINA Todizara todizara.andrianajaina@gmail.com>.
%%
%% Ce ci est le programme principale du document.
%%
%% cecie est le Preambule.tex du document.



%%%%%%%%%%%%%%%%%%%%%%%%%%%%%%%%%%%%%%%%
%           Liste des packages         %
%%%%%%%%%%%%%%%%%%%%%%%%%%%%%%%%%%%%%%%%


%% Faux texte, juste pour la démo
%\usepackage{blindtext}
%%%%%%%%%%%%%%%%%%%%%%%%%%%%%%%%%%%%%%%%%%%%%%%%%%%%%%%%%%%%%%%%%%%%%

%% Réglage des fontes et typo    
\usepackage[utf8]{inputenc}		% LaTeX, comprend les accents !
\usepackage[T1]{fontenc}
\usepackage{float}
\usepackage{graphicx}

%\usepackage[square,sort&compress,sectionbib]{natbib}		% Doit être chargé avant babel
\usepackage[numbers]{natbib}
\usepackage{chapterbib}
	%\renewcommand{\bibsection}{\section{Références}}		% Met les références biblio dans un \section (au lieu de \section*)
		
\usepackage[frenchb]{babel}
\usepackage{lmodern}
\usepackage{ae,aecompl}						% Utilisation des fontes vectorielles modernes
\usepackage[upright]{fourier}



%%%%%%%%%%%%%%%%%%%%%%%%%%%%%%%%%%%%%%%%%%%%%%%%%%%%%%%%%%%%%%%%%%%%%

%% Apparence globale      
     
\usepackage[top=2.5cm, bottom=2cm, left=3cm, right=2.5cm,
			headheight=15pt]{geometry} 
\usepackage{fancyhdr}			% Entête et pieds de page
	\pagestyle{fancy}			% Indique que le style de la page sera justement fancy
	\lfoot[\thepage]{} %gauche du pied de page
	\cfoot{} %milieu du pied de page
	\rfoot[]{\thepage} %droite du pied de page
	\fancyhead[RO, LE] {}	
\usepackage{enumerate}
\usepackage{enumitem}
\usepackage[section]{placeins}	% Place un FloatBarrier à chaque nouvelle section
\makeatletter% même chose pour les \subsection que le package ne gère pas
\renewcommand\subsection{\FloatBarrier\@startsection{subsection}{2}{\z@}
{-3.25ex\@plus -1ex \@minus -.2ex}
{1.5ex \@plus .2ex}{\normalfont\large\bfseries}}
\makeatother
\usepackage{epigraph}
\usepackage[font={small}]{caption}
\usepackage[francais]{minitoc}		% Mini table des matières, en français
	\setcounter{minitocdepth}{2}	% Mini-toc détaillées (sections/sous-sections)
\usepackage{pdflscape}				% Permet d'utiliser des pages au format paysage

%%%%%%%%%%%%%%%%%%%%%%%%%%%%%%%%%%%%%%%%%%%%%%%%%%%%%%%%%%%%%%%%%%%%%

%% Maths                         
\usepackage{amsmath}			% Permet de taper des formules mathématiques
\usepackage{amssymb}			% Permet d'utiliser des symboles mathématiques
\usepackage{amsfonts}			% Permet d'utiliser des polices mathématiques
\usepackage{nicefrac}
\usepackage{upgreek}			% For roman (i.e. upright) lowercase Greek characters

%%%%%%%%%%%%%%%%%%%%%%%%%%%%%%%%%%%%%%%%%%%%%%%%%%%%%%%%%%%%%%%%%%%%%
%%Profondeur titre

\renewcommand\theparagraph{\alph{paragraph})}
\setcounter{secnumdepth}{4}
\setcounter{tocdepth}{4}
\usepackage{shorttoc}%insertion sommaire et table de matiere 
%%%%%%%%%%%%%%%%%%%%%%%%%%%%%%%%%%%%%%%%%%%%%%%%%%%%%%%%%%%%%%%%%%%%%


%% Tableaux
\usepackage{multirow}
\usepackage{booktabs}
\usepackage{colortbl}
\usepackage{tabularx}
\usepackage{multirow}
\usepackage{threeparttable}
\usepackage{etoolbox}
	\appto\TPTnoteSettings{\footnotesize}
\addto\captionsfrench{\def\tablename{{\textsc{Tableau}}}}	% Renome 'table' en 'tableau'

            
            
%%%%%
%%package codage 
\usepackage{listings}
\lstset{
	language=C,
	basicstyle=\footnotesize,
	numbers=left,
	numberstyle=\normalsize,
	numbersep=7pt,
}
%%%%%%%%%%%%%%%%%%%%%%%%%%%%%%%%%%%%%%%%%%%%%%%%%%%%%%%%%%%%%%%%%%%%%
%% Graphiques                    
\usepackage{graphicx}			% Permet l'inclusion d'images
\usepackage{subcaption}
\usepackage{pdfpages}
\usepackage{rotating}
\usepackage{pgfplots}
	\usepgfplotslibrary{groupplots}
\usepackage{tikz}
	\usetikzlibrary{backgrounds,automata}
	\pgfplotsset{width=7cm,compat=1.3}
	\tikzset{every picture/.style={execute at begin picture={
   		\shorthandoff{:;!?};}
	}}
	\pgfplotsset{every linear axis/.append style={
		/pgf/number format/.cd,
		use comma,
		1000 sep={\,},
	}}
\usepackage{eso-pic}
\usepackage{import}
\usepackage{cclicenses}

%%%%%%%%%%%%%%%%%%%%%%%%%%%%%%%%%%%%%%%%%%%%%%%%%%%%%%%%%%%%%%%%%%%%%
% Biblio                        
\makeatletter
\patchcmd{\BR@backref}{\newblock}{\newblock(page~}{}{}	% Pour les back-references, affiche 'page' au lieu de 'p.'
\patchcmd{\BR@backref}{\par}{)\par}{}{}
\makeatother
	
	
%%%%%%%%%%%%%%%%%%%%%%%%%%%%%%%%%%%%%%%%%%%%%%%%%%%%%%%%%%%%%%%%%%%%%
%% Navigation dans le document   
%\usepackage[pdftex,pdfborder={0 0 0},
%			colorlinks=true,
%			linkcolor=black,
%			citecolor=black,
%			pagebackref=false,
%			]{hyperref} %Créera automatiquement les liens internes au PDF
\usepackage{hyperref} %test


%%%%%%%%%%%%%%%%%%%%%%%%%%%%%%%%%%%%%%%%%%%%%%%%%%%%%%%%%%%%%%%%%%%%%
%% Mise en forme du texte        
\usepackage{xspace}
%\usepackage[load-configurations = abbreviations]\
\usepackage{siunitx}
	\DeclareSIUnit{\MPa}{\mega\pascal}
	\DeclareSIUnit{\micron}{\micro\meter}
	\DeclareSIUnit{\tr}{tr}
	\DeclareSIPostPower\totheM{m}
	\sisetup{
    locale = FR,
    inter-unit-product = \ensuremath{\cdot}, % Utilisation de \ensuremath pour la compatibilité mathématique
    range-phrase = ~\`{a}~,
    range-units = single,
		%locale = FR,
	 	% inter-unit-separator=$\cdot$,
	  	%range-phrase=~\`{a}~,     	% Utilise le tiret court pour dire "de... à"
	  	%range-units=single,  			% Cache l'unité sur la première borne
	  }
\usepackage{chemist}
\usepackage[version=3]{mhchem}
\usepackage{textcomp}
\usepackage{numprint}
\usepackage{array}
\usepackage{makeidx}
\usepackage[acronym,xindy,toc]{glossaries}
	\newglossary[nlg]{notation}{not}{ntn}{Notation} 	% Création d'un type de glossaire 'notation'
	\makeglossaries
	\loadglsentries{Glossaire}	% Utilisation d'un fichier externe pour la définition des entrées (Glossaire.tex)
\usepackage{hyphenat}
	
	
\usepackage{hyphenat}





%%%%%%%%%%%%%%%%%%%%%%%%%%%%%%%%%%%%%%%%%%%%%%%%%%%%%%%%%%%%%%%%%%%%%
%% Compilation

\usepackage{silence}
%
%% Virer les erreur dues à minitoc
\WarningFilter{minitoc(hints)}{W0023}
\WarningFilter{minitoc(hints)}{W0024}
\WarningFilter{minitoc(hints)}{W0028}
\WarningFilter{minitoc(hints)}{W0030}

 


%%%%%%%%%%%%%%%%%%%%%%%%%%%%%%%%%%%%%%%%
%           Page de garde              %
%%%%%%%%%%%%%%%%%%%%%%%%%%%%%%%%%%%%%%%%
\makeatletter
\def\@ecole{}
\newcommand{\ecole}[1]{
  \def\@ecole{#1}
}

\def\@specialite{}
\newcommand{\specialite}[1]{
  \def\@specialite{#1}
}

\def\@departement{}
\newcommand{\departement}[1]{
  \def\@departement{#1}
}

\def\@typeRapport{}
\newcommand{\typeRapport}[1]{
  \def\@typeRapport{#1}
}

\def\@anneeUniversitaire{}
\newcommand{\anneeUniversitaire}[1]{
  \def\@anneeUniversitaire{#1}
}

\def\@nomPromo{}
\newcommand{\nomPromo}[1]{
  \def\@nomPromo{#1}
}

\def\@encadreura{}
\newcommand{\encadreura}[1]{
  \def\@encadreura{#1}
}

\def\@encadreurb{}
\newcommand{\encadreurb}[1]{
  \def\@encadreurb{#1}
}

\def\@encadreurc{}
\newcommand{\encadreurc}[1]{
  \def\@encadreurc{#1}
}

\def\@jurya{}{}{}
\newcommand{\jurya}[3]{
  \def\@jurya{#1	& #2	& #3\\}
}
\def\@juryb{}{}{}
\newcommand{\juryb}[3]{
  \def\@juryb{#1,	& #2	& #3\\}
}
\def\@juryc{}{}{}
\newcommand{\juryc}[3]{
  \def\@juryc{#1,	& #2	& #3\\}
}
\def\@juryd{}{}{}
\newcommand{\juryd}[3]{
  \def\@juryd{#1,	& #2	& #3\\}
}
\def\@jurye{}{}{}
\newcommand{\jurye}[3]{
  \def\@jurye{#1,	& #2	& #3\\}
}
\def\@juryf{}{}{}
\newcommand{\juryf}[3]{
  \def\@juryf{#1,	& #2	& #3\\}
}
\def\@juryg{}{}{}
\newcommand{\juryg}[3]{
  \def\@juryg{#1,	& #2	& #3\\}
}
\def\@juryh{}{}{}
\newcommand{\juryh}[3]{
  \def\@juryh{#1,	& #2	& #3\\}
}
\def\@juryi{}{}{}
\newcommand{\juryi}[3]{
  \def\@juryi{#1,	& #2	& #3\\}
}

\makeatother

\makeatletter

\makeatother
